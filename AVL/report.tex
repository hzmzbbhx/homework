\documentclass[UTF8]{ctexart}
\usepackage{geometry, CJKutf8}
\geometry{margin=1.5cm, vmargin={0pt,1cm}}
\setlength{\topmargin}{-1cm}
\setlength{\paperheight}{29.7cm}
\setlength{\textheight}{25.3cm}

% useful packages.
\usepackage{amsfonts}
\usepackage{amsmath}
\usepackage{amssymb}
\usepackage{amsthm}
\usepackage{enumerate}
\usepackage{graphicx}
\usepackage{multicol}
\usepackage{fancyhdr}
\usepackage{layout}
\usepackage{listings}
\usepackage{float, caption}
\usepackage{xcolor}

\lstset{
    basicstyle=\ttfamily, basewidth=0.5em
}

% some common command
\newcommand{\dif}{\mathrm{d}}
\newcommand{\avg}[1]{\left\langle #1 \right\rangle}
\newcommand{\difFrac}[2]{\frac{\dif #1}{\dif #2}}
\newcommand{\pdfFrac}[2]{\frac{\partial #1}{\partial #2}}
\newcommand{\OFL}{\mathrm{OFL}}
\newcommand{\UFL}{\mathrm{UFL}}
\newcommand{\fl}{\mathrm{fl}}
\newcommand{\op}{\odot}
\newcommand{\Eabs}{E_{\mathrm{abs}}}
\newcommand{\Erel}{E_{\mathrm{rel}}}

\begin{document}

\pagestyle{fancy}
\fancyhead{}
\lhead{陈卓, 3230104877}
\chead{数据结构与算法第六次作业}
\rhead{Nov.10th, 2024}

\section{remove()函数的设计}

\subsection{detachMin()的设计}
新增了一个函数:BinaryNode *detachMin(BinaryNode *\&t);

作用是查找以 t 为根的子树中的最小节点,返回这个节点,并从原子树中删除这个节点。显然,当要删除的节点具有两个子树时,通过这个函数返回的右子树最小节点将代替被删除节点。


具体操作:1.递归地向左子树遍历,寻找最小节点;

2.一旦找到最小节点,它将其从当前树中脱离,把指针 t 更新为最小节点的右子节点;

3.返回被移除的最小节点的指针。


\subsection{remove()的修改}
在原有基础上修改:当满足else if (t->left != nullptr \&\& t->right != nullptr) 条件,即要删除的节点具有两个子树时

1.先通过detachMin(t->right)找到需要删除的节点指针t的右子树最小节点指针minNode,

2.再将minNode的左右子树用t所指节点的左右子树替换

3.将当前t的内容保存在oldNode后再用minNode替换t原先内容

4.最后释放原先t的内存oldNode;

实现细节:

\begin{lstlisting}
void remove(const Comparable &x, BinaryNode * &t) 
{   
    else if (t->left != nullptr && t->right != nullptr) 
    {  
        BinaryNode* minNode = detachMin(t->right);
        BinaryNode *oldNode = t;
        minNode->left = t->left; 
        minNode->right = t->right; 
        t = minNode; 
        delete oldNode; 
    }
}
\end{lstlisting}

\section{补充:平衡调整}
参考code代码,private里增加height成员以方便检查平衡,remove的每次递归中都会实现一次balance

\section{本地time测试}
real    0m1.477s

user    0m1.455s

sys     0m0.010s


\end{document}

%%% Local Variables: 
%%% mode: latex
%%% TeX-master: t
%%% End: 
