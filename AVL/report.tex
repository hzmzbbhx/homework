\documentclass[UTF8]{ctexart}
\usepackage{geometry, CJKutf8}
\geometry{margin=1.5cm, vmargin={0pt,1cm}}
\setlength{\topmargin}{-1cm}
\setlength{\paperheight}{29.7cm}
\setlength{\textheight}{25.3cm}

% useful packages.
\usepackage{amsfonts}
\usepackage{amsmath}
\usepackage{amssymb}
\usepackage{amsthm}
\usepackage{enumerate}
\usepackage{graphicx}
\usepackage{multicol}
\usepackage{fancyhdr}
\usepackage{layout}
\usepackage{listings}
\usepackage{float, caption}
\usepackage{xcolor}

\lstset{
    basicstyle=\ttfamily, basewidth=0.5em
}

% some common command
\newcommand{\dif}{\mathrm{d}}
\newcommand{\avg}[1]{\left\langle #1 \right\rangle}
\newcommand{\difFrac}[2]{\frac{\dif #1}{\dif #2}}
\newcommand{\pdfFrac}[2]{\frac{\partial #1}{\partial #2}}
\newcommand{\OFL}{\mathrm{OFL}}
\newcommand{\UFL}{\mathrm{UFL}}
\newcommand{\fl}{\mathrm{fl}}
\newcommand{\op}{\odot}
\newcommand{\Eabs}{E_{\mathrm{abs}}}
\newcommand{\Erel}{E_{\mathrm{rel}}}

\begin{document}

\pagestyle{fancy}
\fancyhead{}
\lhead{陈卓, 3230104877}
\chead{数据结构与算法第五次作业}
\rhead{Oct.30th, 2024}

\section{remove()函数的设计}

\subsection{detachMin()的设计}
新增了一个函数:AvlNode *detachMin(BinaryNode *\&t);

作用是查找以 t 为根的子树中的最小节点,返回这个节点,并从原子树中删除这个节点。显然,当要删除的节点具有两个子树时,通过这个函数返回的右子树最小节点将代替被删除节点。


具体操作:1.递归地向左子树遍历,寻找最小节点;
2.一旦找到最小节点,它将其从当前树中脱离,把指针 t 更新为最小节点的右子节点;
3.返回被移除的最小节点的指针。


\subsection{remove()的修改}
在原有基础上修改:当满足else if (t->left != nullptr \&\& t->right != nullptr) 条件,即要删除的节点具有两个子树时,先通过detachMin(t->right)找到需要删除的节点指针t的右子树最小节点指针minNode,
再将minNode的左右子树用t所指节点的左右子树替换,将当前t存在oldNode后再用minNode替换,最后释放原先t的内存oldNode;

\section{结果呈现和分析}
创一个int类型Avltree,依次插入10、5、15、3、7、12、18,再依次remove、print 7和10;

如果程序无误,应输出:

Tree after removing 7:

3

5

10

12

15

18

Tree after removing 10:

3

5

12

15

18

\section{检查内存泄漏}

我用 valgrind 进行测试,发现没有发生内存泄露。


\end{document}

%%% Local Variables: 
%%% mode: latex
%%% TeX-master: t
%%% End: 
