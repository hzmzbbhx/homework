\documentclass[UTF8]{ctexart}
\usepackage{geometry, CJKutf8}
\geometry{margin=1.5cm, vmargin={0pt,1cm}}
\setlength{\topmargin}{-1cm}
\setlength{\paperheight}{29.7cm}
\setlength{\textheight}{25.3cm}

% useful packages.
\usepackage{amsfonts}
\usepackage{amsmath}
\usepackage{amssymb}
\usepackage{amsthm}
\usepackage{enumerate}
\usepackage{graphicx}
\usepackage{multicol}
\usepackage{fancyhdr}
\usepackage{layout}
\usepackage{listings}
\usepackage{float, caption}
\usepackage{xcolor}

\lstset{
    basicstyle=\ttfamily, basewidth=0.5em
}

% some common command
\newcommand{\dif}{\mathrm{d}}
\newcommand{\avg}[1]{\left\langle #1 \right\rangle}
\newcommand{\difFrac}[2]{\frac{\dif #1}{\dif #2}}
\newcommand{\pdfFrac}[2]{\frac{\partial #1}{\partial #2}}
\newcommand{\OFL}{\mathrm{OFL}}
\newcommand{\UFL}{\mathrm{UFL}}
\newcommand{\fl}{\mathrm{fl}}
\newcommand{\op}{\odot}
\newcommand{\Eabs}{E_{\mathrm{abs}}}
\newcommand{\Erel}{E_{\mathrm{rel}}}

\begin{document}

\pagestyle{fancy}
\fancyhead{}
\lhead{陈卓, 3230104877}
\chead{四则运算计算器}
\rhead{11.5, 2024}

\section{设计思路}
我们的程序分为几个部分:
\begin{itemize}
    \item 检查输入的合法性,包括字符合法性、运算符连续性、括号匹配、除数不为0等。
    \item 使用后缀表达式转化算法将中缀表达式转成后缀表达式。
    \item 计算后缀表达式的值。
\end{itemize}

\section{代码实现}

\subsection{函数定义}

\subsubsection{操作符优先级函数:youxianji}
\begin{verbatim}
int youxianji(char op) 
{
    if (op == '+' || op == '-') return 1;
    if (op == '*' || op == '/') return 2;
    return 0;
}
\end{verbatim}
该函数用于确定操作符的优先级。加法和减法的优先级为1,乘法和除法的优先级为2。

\subsubsection{执行运算函数:Operation}
\begin{verbatim}
double Operation(double a, double b, char op) 
{
    switch (op) 
    {
        case '+': return a + b;
        case '-': return a - b;
        case '*': return a * b;
        case '/':
            if (b == 0) 
            {
                throw runtime_error("ILLEGAL: Division by zero!");
            }
            return a / b;
        default:
            throw runtime_error("ILLEGAL: Illegal operation!");
    }
}
\end{verbatim}
根据操作符执行相应的数学运算。如果遇到除以零的情况,抛出异常。

\subsubsection{数字转换函数:transformer}
\begin{verbatim}
double transformer(const string& str) 
{
    stringstream ss(str);
    double value;
    ss >> value;
    return value;
}
\end{verbatim}
该函数使用 stringstream 将字符串转换为 double 类型的数值。

\subsection{主函数:evaluate}
\begin{verbatim}
double evaluate(const string& expression) 
{
    stack<double> values; // 用于存储数字
    stack<char> ops;      // 用于存储操作符
\end{verbatim}
主函数开始,定义两个栈,一个用于存储数字,另一个用于存储操作符。

\begin{verbatim}
    char lastChar = ' ';
    for (int i = 0; i < expression.length(); i++) 
    {
        if (expression[i] == ' ') continue;
\end{verbatim}
初始化一个字符变量 `lastChar` 记录前一个字符,并遍历整个表达式。

\begin{verbatim}
        if (isdigit(expression[i]) || expression[i] == '.') 
        {
            string number;
            while (i < expression.length() &&
                   (isdigit(expression[i]) || expression[i] == '.' || expression[i] == 'e' || expression[i] == 'E')) 
            {
                number += expression[i++];
            }
            i--; // 因为for循环会自增,所以需要减去1
            values.push(transformer(number));
            lastChar = 'n';
        }
\end{verbatim}
处理数字,包括整数和小数。将数字字符串转换为数值并压入 `values` 栈。

\begin{verbatim}
        else if (expression[i] == '(') 
        {
            ops.push(expression[i]);
            lastChar = '(';
        }
        else if (expression[i] == ')') 
        {
            while (!ops.empty() && ops.top() != '(') 
            {
                double val2 = values.top(); values.pop();
                double val1 = values.top(); values.pop();
                char op = ops.top(); ops.pop();
                values.push(Operation(val1, val2, op));
            }
            if (ops.empty() || ops.top() != '(') 
            {
                throw runtime_error("ILLEGAL: Mismatched parentheses!");
            }
            ops.pop();
            lastChar = ')';
        }
\end{verbatim}
处理括号,左括号直接压栈,右括号则弹出操作符并计算,直到遇到左括号,如果括号不匹配,抛出异常。

\begin{verbatim}
        else if (expression[i] == '+' || expression[i] == '-' || expression[i] == '*' || expression[i] == '/') 
        {   
            if (expression[i] == '-') 
            {
                if (lastChar == ' ' || lastChar == '(' || lastChar == '+' || lastChar == '-' || lastChar == '*' || lastChar == '/') 
                {
                    i++; 
                    string number;
                    while (i < expression.length() &&
                        (isdigit(expression[i]) || expression[i] == '.' || expression[i] == 'e' || expression[i] == 'E')) 
                    {
                        number += expression[i++];
                    }
                    i--; 
                    values.push(-transformer(number)); 
                    lastChar = 'n'; 
                    continue; 
                }
            }

            if (lastChar == '+' || lastChar == '-' || lastChar == '*' || lastChar == '/') 
            {
                throw runtime_error("ILLEGAL: Invalid expression! Operator cannot follow operator.");
            }
            while (!ops.empty() && youxianji(ops.top()) >= youxianji(expression[i])) 
            {
                double val2 = values.top(); values.pop();
                double val1 = values.top(); values.pop();
                char op = ops.top(); ops.pop();
                values.push(Operation(val1, val2, op));
            }
            ops.push(expression[i]);
            lastChar = expression[i];
        }
\end{verbatim}
首先检查当前字符是否为负号,如果是,并且前一个字符是空格、左括号或者是一个操作符,那么我们可以确定这是一个负数。代码将跳过负号,并开始收集数字字符,直到遇到非数字字符。收集到的数字字符串将被转换为数值,并取反后压入数值栈中。然后,更新 `lastChar` 为 'n',表示前一个字符是一个数字,并继续处理下一个字符

如果不是,判断lastChar是否为字符,如果是的话抛出异常。

\begin{verbatim}
        else 
        {
            throw runtime_error("ILLEGAL: Invalid character");
        }
    }
\end{verbatim}
如果遇到非法字符,则抛出异常。

\begin{verbatim}
    while (!ops.empty()) 
    {
        double val2 = values.top(); values.pop();
        double val1 = values.top(); values.pop();
        char op = ops.top(); ops.pop();
        values.push(Operation(val1, val2, op));
    }
    return values.top();
\end{verbatim}
最后,计算剩余的操作符。


\section{测试用例及结果}
以下是所测试的表达式及其对应的计算结果:

\begin{center}
\begin{tabular}{|l|l|}
\hline
\textbf{表达式} & \textbf{计算结果} \\
\hline
$((3.14 + 2.71) * (1.41 - 0.41)) + (5.0 / 2.0)$ & $8.3500$ \\
\hline
$(10.0 - (3.5 * 4.2)) + (15.6 / 2.1)$ & $2.7286$ \\
\hline
$((8.0 / 2.0) * (3.0 + 2.0)) - (7.2 - 4.5)$ & $17.3000$ \\
\hline
$(5.5 + (3.3 * 2.2)) / (1.1 * 9.0)$ & $1.2889$ \\
\hline
$((2.0 * (1.0 + 5.0)) / (3.0 - 2.0)) + (9.9 * 0.1)$ & $12.9900$ \\
\hline
$(4.5 * (3.3 - 2.2)) + ((5.5 - 1.5) / 2.6)$ & $6.4885$ \\
\hline
$((2.5 + 3.0) * (5.0 - 1.0)) + (0.5 * 7.5)$ & $25.7500$ \\
\hline
$((9.9 / 3.3) - 1.0) * (2.0 + 4.0)$ & $12.0000$ \\
\hline
$(((8.0 + 2.5) / 2.5) * (3.0 - 0.9)) + 5.0$ & $13.8200$ \\
\hline
$(10.0 - (2.0 * (3.0 + 7.0))) / (2.0 * 2.0)$ & $-2.5000$ \\
\hline
$1e5 + 1 $ & $100001.0000$ \\
\hline
$1 + - 1 $ & $0.0000$ \\
\hline
$-1 + 2 $ & $1.0000$ \\
\hline
$1 / 0 $ & $- IILEGAL: Division by zero!$ \\
\hline
$1 + + 1 $ & $- IILEGAL: Invalid expression! Operator cannot follow operator.$ \\
\hline
$1 + 2)$ & $- IILEGAL: Mismatched parentheses!$ \\
\hline
$? $ & $- IILEGAL: Invalid character$ \\
\hline
\end{tabular}
\end{center}
用例皆正确无误,符合预期。

如需新增用例,可在main.cpp中的expressions数组中添加。

\end{document}

%%% Local Variables: 
%%% mode: latex
%%% TeX-master: t
%%% End: 
