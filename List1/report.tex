\documentclass[UTF8]{ctexart}
\usepackage{geometry, CJKutf8}
\geometry{margin=1.5cm, vmargin={0pt,1cm}}
\setlength{\topmargin}{-1cm}
\setlength{\paperheight}{29.7cm}
\setlength{\textheight}{25.3cm}

% useful packages.
\usepackage{amsfonts}
\usepackage{amsmath}
\usepackage{amssymb}
\usepackage{amsthm}
\usepackage{enumerate}
\usepackage{graphicx}
\usepackage{multicol}
\usepackage{fancyhdr}
\usepackage{layout}
\usepackage{listings}
\usepackage{float, caption}
\usepackage{xcolor}

\lstset{
    basicstyle=\ttfamily, basewidth=0.5em
}

% some common command
\newcommand{\dif}{\mathrm{d}}
\newcommand{\avg}[1]{\left\langle #1 \right\rangle}
\newcommand{\difFrac}[2]{\frac{\dif #1}{\dif #2}}
\newcommand{\pdfFrac}[2]{\frac{\partial #1}{\partial #2}}
\newcommand{\OFL}{\mathrm{OFL}}
\newcommand{\UFL}{\mathrm{UFL}}
\newcommand{\fl}{\mathrm{fl}}
\newcommand{\op}{\odot}
\newcommand{\Eabs}{E_{\mathrm{abs}}}
\newcommand{\Erel}{E_{\mathrm{rel}}}

\begin{document}

\pagestyle{fancy}
\fancyhead{}
\lhead{陈卓, 3230104877}
\chead{数据结构与算法第四次作业}
\rhead{Oct.16th, 2024}

\section{测试程序的设计思路}
\textcolor{red}{备注}:设计了一个 \texttt{print()} 在 \texttt{List} 的 public 部分,用来打印 \texttt{List} 的元素,方便检验各个函数的有效性。  

\subsection{push\_back、pop\_back、pop\_front、push\_front(10)的验证}
我首先创建了一个 \texttt{int} 类型的空链表 \texttt{lst1},再用 \texttt{push\_back()} 插入 0 到 9 的元素。  
接下来,使用 \texttt{pop\_back()}、\texttt{pop\_front()} 以及 \texttt{push\_front(10)} 删除列表最前和最后的元素 0 和 9,并从最前面插入 10。  
最后打印 \texttt{lst1} 的元素,如果函数无误,应输出: 

\textcolor{red}{10 1 2 3 4 5 6 7 8}

\subsection{big5的验证}
通过 \texttt{lst2}、\texttt{lst3}、\texttt{lst4}、\texttt{lst5}、\texttt{lst6} 分别验证构造函数、复制构造函数、移动构造函数、复制赋值运算符以及移动赋值运算符,参数设置为 \{1,2,3,4,5\},类型为 \texttt{int}。  
(最后两个实际上是一个函数 \texttt{List \&operator=(List copy)})  
依次打印验证,如果函数无误,应输出:

\textcolor{red}{1 2 3 4 5} 

\textcolor{red}{1 2 3 4 5} 

\textcolor{red}{1 2 3 4 5} 

\textcolor{red}{1 2 3 4 5} 

\textcolor{red}{1 2 3 4 5}

\subsection{size、empty、clear的验证}
沿用 \texttt{lst1},用 \texttt{size()} 输出 \texttt{lst1.theSize},用 \texttt{empty()} 检查 \texttt{lst1} 是否为空,再用 \texttt{clear()} 清空 \texttt{lst1},最后检查是否成功清空。  
如果函数无误,应输出: 

\textcolor{red}{9}  
\textcolor{red}{0}  
\textcolor{red}{1}

\subsection{back、front的验证}
\texttt{lst1} 清空后将 \texttt{lst6} 复制给 \texttt{lst1},输出尾元素 \texttt{back()} 和首元素 \texttt{front()},同时验证其可修改性,方便起见lst1.front()=1,理论上不应报错;。  
如果函数无误,应输出:

\textcolor{red}{5 1}

\subsection{insert、erase的验证}
令迭代器 \texttt{it1} 指向 \texttt{lst1} 第二个元素,并验证并 insert(\texttt{it1}, 11); 

令迭代器 \texttt{it2} 指向 \texttt{lst1} 最后一个元素,并验证 \texttt{erase(it2)}; 

令迭代器 \texttt{it3} 指向 \texttt{lst1} 最后一个元素,使用 \texttt{erase(it1, it3)} 删除该范围节点;

令迭代器 \texttt{it4} 指向 \texttt{lst1} 第一个元素,使用 \texttt{*it4 = 18} 验证迭代器 \texttt{begin()} 函数的可修改性,\texttt{end()} 同理。  

如果函数无误,应输出: 

\textcolor{red}{1 11 2 3 4 5} 

\textcolor{red}{1 11 2 3 4} 

\textcolor{red}{1 11 4} 

\textcolor{red}{18 11 4}


\section{测试的结果}

测试结果一切正常(上面红色部分)。

我用 valgrind 进行测试,发现没有发生内存泄露。

\section{(可选)bug报告}

我发现了一个 bug,触发条件如下:

\begin{enumerate}
    \item 首先……\qquad List为空时.可以clear()
    \item 然后…… \qquad 进行erase操作 
    \item 此时发现……报错:\textcolor{red}{Segmentation fault}
\end{enumerate}


据我分析,它出现的原因是:it5.current为nullpur,没有next和prev,无法完成erase

\end{document}

%%% Local Variables: 
%%% mode: latex
%%% TeX-master: t
%%% End: 
